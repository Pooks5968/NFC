\documentclass[12pt]{article}

\usepackage[margin=1in]{geometry}

\begin{document}
\title{Defense Against W NFC}
\author{Max Feldman \and Stephanie Rogers \and Richard Xia}
\maketitle

\section{Abstract}
1-3 paragraph version of introduction

\section{Introduction}
1-2 page version of paper

Related NFC Technology and Security - authenticating tags rather than users
* MiFare: hardware that allows the locking of data and allows symmetric mutual authentication of users and write protection to defend against reading and writing of the tag (privacy)
* FeliCa: authentication and write protection 
* AFAICT: lock the data (not overwritable), but no authentication!

\section{Background}
describe NFC

Android/permission system

intents

* current apps, what permissions do they use
** characteristics of apps
*** tiny
*** superapps

* threat model
** writing sensitive data
** malicious tags which cause automatic running of apps which can cost money
** eavesdropping
** malware

\section{Related Work}

\section{Analysis of Apps}
* Why analyze
** general coding practices re: NFC
** what practices contribute to vulnerabilities of NFC apps
** As for as we know, no broad analysis of NFC app level stuff
** provide suggestion for NFC app devs (we are in a unique position in that this tech is up and coming)
* static
** trace where data read from NFC tags is ended
** other static tools?
** Beam
** Use of crypto libs
** Overwriting tags- do apps in general 0 out prior data, or not?

* dynamic
** taintdroid
** droidblaze

* manual
** looked at source code
** played with apps
** trace exec paths if possible
** BEAM
*** If you scan an app, it'll take you to the app store
* permission scraping etc

\section{Proposed Solution}
We propose a novel solution to the aforementioned issues which provides users 

\TODO{write all of this}

talking points:
trust model
authentication
key management
usability

* Goals/non-goals
* architecture
** distributed authentication
*** generalizable to arbitrary inputs to phone?

* Goals

As a user
I want to be able to verify a tag came from a given author
In order to ensure I am not acting on malicious inputs

As a user
I want to be able to keep track of tag authors
In order to expand the scope of tags I can trust

As a user
I want to be able to verify my own tags
In order to ensure use my own tags

As a user
I want to be able to encrypt my own tags
In order to ensure no one else can read my tags

As a user
I want to keep the identities of trusted tag authors correct
In case if good authors go bad

As a user
I want to report malicious tag authors
To warn other users of bad authors

As a tag author
I want to share my key with a user
To prevent others from impersonating me

** integrity
** authentication

* Non goals
** confidentiality of tag data
** ignoring malicious applications
** no central authority
** not ensuring security of key exchange


How to get keys?
* Go to website of tag author you trust, DL their key
* see it for the first time (good info in prompt?)
* add a key repo
* NFC tag (P2P sign)
* No signature: different prompt

Prompts:
* disable all pmopts
* never allow this key?
* allow key once
* allow key always
* allow all keys

* We should alway state why we make any choice

\section{Implementation}
We implemented our authentication system as an Android library that other Android applications can include into their own projects.
We developed a special tag interface class we call \texttt{SignedNdef} which encapsulates all communication with our secure NDEF tags.
\texttt{SignedNdef} expects a tag to contain an extra record at the end of an NDEF message which contains a signature of the other NDEF records.
Android developers can use the API we developed for interfacing with a \texttt{SignedNdef} object, which we designed to closely mimic the existing official Android \texttt{Ndef}\footnote{http://developer.android.com/reference/android/nfc/tech/Ndef.html} object for reading and writing NDEF tags.

On instantiation, a \texttt{SignedNdef} object will attempt to open up the list of authorized keys saved on the Android device.
It provides methods for opening, closing, reading, and writing NDEF-formatted tags by delegating to a private instance of the \texttt{Ndef} class.
In addition, it will automatically enforce the security protocols described earlier.
When reading, it will query the \texttt{KeyManager} to find a public key which matches the signature of the

* implementation
** proof of concept
** recommendations
* key file
* key add/remove
* warnings
* (a duplicate app w/ some functionality but no auth, for demo)
* API
* warning for invalid signature

\section{Evaluation}
* simplicity/overhead for user (warning message?)
* how many apps can benefit from this?
* applicability
** do 3 case studies (examples of applying it to an application)
** do it on github
* How many additional lines of code required for using API?

* (performance (time+space))
* security of this scheme
** bits of security (e.g. signature size)
** assumptions/foundations (e.g. secure key distribution)
** key storage on Android phones can't be tampered with

* LIMITATIONS

* key add/revoke
** app perm to add
** app perm to remove
** done via syscalls, i.e. android holds this
*** some intent?
** some way for app developer to add centralized repo of keys
** automatic get key

* Does phone already have a pub key pair?
* backwards compatibility
** Read NDEF standards and how Android handles them

brainstorm                         works for                           bad
- NDEF record/type for signature   updated os
                                   app viewing finite \# records
- length exclused signature        above                               violates standards
  + above                          app viewing variable \# of records   

\section{Discussion}

\section{Future Work}
* user study, human testing
* overhead imposed by warnings/approvals
** delves into usable security
* compression/optimization
* OS perms, kernel mods to support this

\section{Conclusion}

\section{Acknowledgements}

\section{References}

\end{document}
