\documentclass[12pt]{article}

\usepackage[margin=1in]{geometry}
\usepackage{xcolor}

\newcommand\TODO[1]{\textcolor{red}{#1}}

\begin{document}
\title{Defense Against W NFC}
\author{Max Feldman \and Stephanie Rogers \and Richard Xia}
\maketitle

\section{Abstract}
1-3 paragraph version of introduction

\section{Introduction}
1-2 page version of paper



Related NFC Technology and Security - authenticating tags rather than users
* MiFare: hardware that allows the locking of data and allows symmetric mutual authentication of users and write protection to defend against reading and writing of the tag (privacy)
* FeliCa: authentication and write protection 
* AFAICT: lock the data (not overwritable), but no authentication!

\section{Background}
Near Field Communication (NFC) is an emerging technology for wireless communication which allows NFC-enabled devices to transfer small amounts of data at a close proximity, usually no more than a few centimeters. NFC offers current and predicted functionality of contactless transactions, data exchange and an alternative to traditional contactless technologies such as RFID and QR codes. Specifically aimed at mobile phones, devices have the ability to communicate with other NFC-enabled mobile phones to transfer data peer-to-peer in active mode or read information stored on NFC tags and cards in passive mode. 

\subsection{Android NFC}

The Android system sends and receives NFC data in the form of NDEF (NFC Data Exchange Format) messages. These messages, which are stored in the NFC tag, act as a container for one or more NDEF Records which contain typed data such as MIME-type media, a URI or a custom application payload (plagarism).  

Specifically, the Android system offers the ability to read and write NFC tags for passive mode, and beam NDEF messages from one device to another with Android Beam$^{2}$ for active mode. 

\subsection{Android Intents}
The core components of an Android application are activated through messages called intents.$^{1}$ Intents are used for inter-application and intra-application communication by sending messages with relevant information about operations to be performed to various applications.  

When NFC is enabled and the screen is unlocked, Android devices will always be searching for input from an NFC tag or device. When the device discovers an NFC tag, an intent is sent to the phone with information including which application is most appropriate to handle the data contained in the NFC tag. This information is processed by Android's special tag dispatch system to determine which activity to launch. It turns out that through NFC one can force some phones to parse images, videos, contacts, office documents, call arbitrary phone numbers and even open up web pages in the browser, all without user interaction.  

\subsection{Android Permission System}
Android applications run in a sandbox which allows areas of the system to be isolated and thus limit access to the security-related parts of the Android API. However, access to these resources can be granted to an application if the developer requests the appropriate permissions in the application's manifest. Android developers are expected to use least-privilege with their permission requests: limit the permissions requested to only those that are absolutely necessary for the function of their applications. By analyzing the use of permissions, one can learn about the potential vulnerabilities introduced in an application. 

\subsection{Application Characteristics}
*** tiny
*** superapps

\subsection{Threat Model}
As with any new technology, comes the potential for vulnerabilities within the system. Our paper attempts to analyze the potential threats that this new technology may introduce and how the coding practices of NFC applications relate to these threats. We focus predominantly on the ability of an attacker to force a device to scan a malicious tag. However, we consider other threat models including reading a tag in which sensitive data was written to, eavesdropping on the data transferred over the NFC connection, and creating a malicious NFC application. 

\subsubsection{Malicious Tags}
It is possible to have a device scan a tag and perform the action specified by the tag without any user interaction, as mentioned previously. If an attacker can scan a tag unnoticed by the user, then the attacker can force the device to perform an arbitrary NFC action simply by writing it as part of a malicious tag. This opens the opportunity to have a phone open an arbitrary web url in the browser or call a specified phone number. 

If the url is malicious in any way, opening it in the browser can cause... \TODO{issues should be talked about here. Talk about how the malicious application might have access to the sensitive information on the phone itself or corrupt the phone in some way. I am not entirely sure what to say here.} 

If the attacker has the ability to automatically have the phone call a specified number, then a tolled number could easily be inserted and thus result in a monetary loss on the part of the user. This is a direct and obvious vulnerability in any application the calls an arbitrary phone number specified by a tag. An example of an application that does so is Samsung's Tectile application which, when scanning a tag with a phone number and directions to call this number, will do so without any further consideration. 

\subsubsection{Writing Sensitive Data}
Several NFC applications have the possibility of writing information to an NFC tag. This means that users have the ability to write anything they want to their own tags. It may be beneficial for some users to use these tag writing applications to write sensitive or critical information to tags they keep in specific locations. \TODO{An example of this might be... }. If the attacker was ever able to gain access to these tags, there is no security stopping him/her from reading the tag and viewing the contents with the sensitive data. 

\subsubsection{Eavesdropping}


\subsubsection{Malware}
The last threat model we consider is that of an attacker creating his/her own malicious application. Given the fact that a majority of users ignore the permissions requested in a given application, an NFC application could easily request unnecessary and misleading permissions that give access to sensitive information from the phone itself. The application could then have access to such things as the contacts list, phone data, emails, and more. If the application also had some way to relay this information back to the attacker, for example, permission to access the Internet, then all of the user's information would be compromised. Since several NFC applications require multiple permissions, a user might not be as easily able to discern the necessary permissions or understand what the permission is being used for, thus making it more likely for the user to simply trust the application. However, this is possible for applications outside of NFC specific ones, and thus we choose not to focus on this attack. 

\TODO{How is this different than any other application? Talk more about that. More NFC applications need certain permissions than normal applications, so a user giving access to certain resources might be more likely?}
\\
\\
1. http://developer.android.com/guide/components/intents-filters.html
\\
2. http://developer.android.com/guide/topics/connectivity/nfc/nfc.html


\section{Related Work}
%Close range radio wave communication has existed for some time, but the application of NFC in mobile phones is a recent development, and still growing in popularity. The NFC forum (http://www.nfc-forum.org/), which publishes NFC standards (http://www.nfc-forum.org/specs/spec_list/) and best practices, was only formed in 2004. NFC also relies on the interface and protocol specification (http://www.iso.org/iso/catalogue_detail.htm?csnumber=38578) published in 2004. The relative youth of this field has resulted in many aspects remaining largely unexplored. In this section we present relevant past work in the area of NFC security, as well as mobile phone application security in general.
%
%One major foray into the analysis of application-level NFC vulnerabilities was presented by Charlie Miller at BlackHat 2012 (http://media.blackhat.com/bh-us-12/Briefings/C_Miller/BH_US_12_Miller_NFC_attack_surface_WP.pdf). Miller employs fuzz testing to evaluate the security of both the NFC protocol layer implemented on various mobile phones, and subsequently to evaluate application-level security. Miller discusses the successes of fuzzing in identifying various NFC messages which will crash a device. Further, Miller provides evidence that carefully crafted NFC payloads can do much more than just crash a device- in some cases phones can be forced to parse arbitrary data or open web pages without user interaction. Miller’s work provides a wide analysis of the NFC attack surface, but our research will seek to provide much more depth in the exploration of application-level vulnerabilities.
%
%Kortvedts’s “Securing Near Field Communications” (http://ntnu.diva-portal.org/smash/get/diva2:347744/FULLTEXT01) is a precursor to Miller’s Blackhat presentation, and presents a view of NFC vulnerabilities and privacy concerns which remains largely unchanged since the paper’s publication. Kortvedt explains means of intercepting NFC data, even outside of the advertised 10 cm range, as well as existing attempts at providing a cryptographically secure interface for NFC communication. Some applications employ cryptography (the majority being payment applications), but no current secure-NFC framework enjoys widespread use. Kortvedt proposes a framework which attempts to provide solutions to several issues, but from experience it has not gained widespread adoption. We also seek to analyze the issue of providing a simple, secure NFC API for casual and inexperienced application developers.
%
%RFID, a precursor to and superset of NFC, has also raised several security issues in the past. Though RFID is a much older technology, it is remains relevant as both the foundation for NFC, and as a technology which is still employed in various applications (most prominently touch-activated credit cards). Heydt-Benjamin et al. (http://people.cs.umass.edu/~kevinfu/papers/RFID-CC-LNCS.pdf) discuss issues involved in early generation deployments of RFID credit cards, and were successfully able to initiate various attacks on three major RFID enabled credit cards (including attacks which access private information or enable arbitrary purchases by the attacker). Such attacks remain important considerations due to the growing prominence of smartphone payment applications  (such as Google Wallet), so we keep all of these vulnerabilities in mind as we explore NFC security vulnerabilities. This paper presents some countermeasures, but some may be impractical (such as shielding) and others can be improved upon due to more flexible computing options of smartphones.
%
%"Security in NFC, Strengths and Weaknesses" (http://events.iaik.tugraz.at/RFIDSec06/Program/papers/002%20-%20Security%20in%20NFC.pdf) discusses general security issues with NFC (beyond the scope of just mobile phones). The paper discusses threats of eavesdropping, data corruption, data modification, data insertion, and man-in-the-middle attacks (though MITM is dismissed as practically impossible). The paper recommends establishing a secure channel between devices in order to prevent such attacks, and provides valuable discussion of how this channel would work. The paper does not, however, provide an implementation (as the purpose of the paper was to illustrate general NFC issues). As our focus is restricted to Android devices, we have the ability to recommend and implement Android-specific solutions.
%
%Francis et al. (http://eprint.iacr.org/2011/618.pdf) discuss the potential for relay attacks on NFC transactions, and implement a software version of this attack which can be used on an NFC-enabled device. A relay attack exploits the assumption that two devices engaged in an NFC transaction are actually adjacent (by placing a proxy in between the two devices). This is a serious potential exploit which must be defended against by any security-conscious application (as it may allow for arbitrary credit card purchases, for example). The paper suggests several potential counters to such attacks; our analysis examines how popular applications employ relay-attack countermeasures, and to what degree they are effective.
%
%"Using QR tags to Attack SmartPhones (Attaging)" presents a discussion of potential attacks which leverage scanning QR codes (Which can then open malicious links). The attacks made possible by malicious QR codes may in many cases be possible via NFC; some NFC applications will open arbitrary links transmitted, or load other content. An attacker could provide a malicious NFC tag which causes the user's phone to navigate to a malicious link (the user may not even be aware of this action, as was previously discussed). In addition to new potential exploits and leakage of private data resulting from NFC applications, previously encountered attacks may also be possible.
%
%Analysis of smartphone applications in general (though we restrict our focus to Android devices, as iPhones do not yet offer NFC) is a much more thoroughly explored field, and several projects provide valuable insight into application testing.
%
%TaintDroid (http://static.usenix.org/event/osdi10/tech/full_papers/Enck.pdf), in order to evaluate privacy leakage, surveyed the 50 most popular free applications on the Android marketplace. This approach provided valuable data regarding how private data is handled by common Android applications, and presents a good technique for large-scale application evaluation. Surveying popular NFC applications will be an important facet of our project, and the TaintDroid approach was able to gather highly representative data of the application area. We intended to leverage a similar approach in order to gather useful data regarding common NFC usage and potential leakage of information over NFC channels.
%
%There are several other tools for both static and dynamic analysis of Android applications, including Droidbox (http://code.google.com/p/droidbox/), which attempts to provide thorough analysis of an application, including its information leakage, data sent over the network, and use of Android cryptography operations, for example. Analysis such as this is valuable for determining whether or not an application even attempts to prevent information leakage when NFC transmission is used.
%
%While RFID technology and Android application analysis have both been explored, there has been very limited exploration of Android-specific NFC applications, and their potential vulnerabilities. We leverage this previous exploration to provide large-scale analysis of NFC applications, vulnerabilities, and coding practices, as well as to recommend best practices for preventing NFC attacks.

\section{Analysis of Apps}
* Why analyze
** general coding practices re: NFC
** what practices contribute to vulnerabilities of NFC apps
** As for as we know, no broad analysis of NFC app level stuff
** provide suggestion for NFC app devs (we are in a unique position in that this tech is up and coming)
* static
** trace where data read from NFC tags is ended
** other static tools?
** Beam
** Use of crypto libs
** Overwriting tags- do apps in general 0 out prior data, or not?

* current apps, what permissions do they use
<ML Classifier here>

* dynamic
** taintdroid
** droidblaze

* manual
** looked at source code
** played with apps
** trace exec paths if possible
** BEAM
*** If you scan an app, it'll take you to the app store
* permission scraping etc

\section{Proposed Solution}
We propose a novel solution to the aforementioned issues which provides users 

\TODO{write all of this}

talking points:
trust model
authentication
key management
usability

* Goals/non-goals
* architecture
** distributed authentication
*** generalizable to arbitrary inputs to phone?

* Goals

As a user
I want to be able to verify a tag came from a given author
In order to ensure I am not acting on malicious inputs

As a user
I want to be able to keep track of tag authors
In order to expand the scope of tags I can trust

As a user
I want to be able to verify my own tags
In order to ensure use my own tags

As a user
I want to be able to encrypt my own tags
In order to ensure no one else can read my tags

As a user
I want to keep the identities of trusted tag authors correct
In case if good authors go bad

As a user
I want to report malicious tag authors
To warn other users of bad authors

As a tag author
I want to share my key with a user
To prevent others from impersonating me

** integrity
** authentication

* Non goals
** confidentiality of tag data
** ignoring malicious applications
** no central authority
** not ensuring security of key exchange


How to get keys?
* Go to website of tag author you trust, DL their key
* see it for the first time (good info in prompt?)
* add a key repo
* NFC tag (P2P sign)
* No signature: different prompt

Prompts:
* disable all pmopts
* never allow this key?
* allow key once
* allow key always
* allow all keys

* We should alway state why we make any choice

\section{Implementation}
We implemented our authentication system as an Android library that Android developers can include and access in their projects.
Our current implementation provides both means of writing and reading tags using our authentication scheme, allowing developers of both tag-reading and tag-writing applications to benefit from increased security.
We present this implementation as a proof-of-concept; ideally, such additions to the Android NFC stack would be implemented at OS-level.
There are several advantages to kernel modifications, which are discussed below.

\subsection{SignedNdef}
The Android NFC API provides access to \texttt{Ndef}, which allows access to the NDEF message contained on a tag.
Our API offers \texttt{SignedNdef}, which offers the same methods as \texttt{Ndef}, but implicitly handles authentication.
Android developers can use the API we developed for interfacing with a \texttt{SignedNdef} object, which we designed to closely mimic the existing official Android \texttt{Ndef}\footnote{http://developer.android.com/reference/android/nfc/tech/Ndef.html} object for reading and writing NDEF tags.
The approach we use for authentication is to use the Digital Signature Algorithm (DSA) to sign the contents of a message, thus allowing the reader of a tag to verify that the tag was written by a trusted tag author.
Our data format for secure NDEF tags appends an extra NDEF record to the end of an NDEF message.
This record contains both information regarding the public key used for signing (discussed in further detail below) and the DSA signature of the original Ndef message.
\TODO{Show figure for the new record format}

\subsection{Key Management}
Users need to be able to add and remove keys from their base of trusted authors (either manually, or through a tag reader application).
For our API, we use Android's \texttt{SharedPreferences}, which provides persistent key-value storage.
Our current implementation provides a unique instance of this key storage (a \texttt{KeyFile}) to an individual application (ie. a given application can only access and modify its own keys).
Key sharing between applications is a useful feature which would be simpler to implement securely at OS-level.
When a user reads a tag, the tag's signature record will either contain a public key in its entirety, or contain a URI pointer to a public key.
If this public key has been added to the Key File, it is considered trusted, otherwise, the user must add this key.
\TODO{``tag providers''? or ``tag authors''? what should we name them?}
When a user writes a tag using this API, the private key for the phone is accessed and a DSA signature is generated and appended.

\subsection{Application Flow}
On instantiation, a \texttt{SignedNdef} object will attempt to open up the list of authorized keys saved on the Android device.
It provides methods for opening, closing, reading, and writing NDEF-formatted tags by delegating to a private instance of the \texttt{Ndef} class.
In addition, it will automatically enforce the security protocols described earlier.
When reading a tag, it will query the \texttt{KeyManager} to find a public key which matches the signature of the signed Ndef tag.
If it can locate a valid public key which matches the signature, it will transparently return the non-signature records on the tag.
In case of any errors or failure to locate a valid public key, it will throw an exception.
\TODO{include figure for application flow}


* implementation
** proof of concept
** recommendations
* key file
* key add/remove
* warnings
* (a duplicate app w/ some functionality but no auth, for demo)
* API
* warning for invalid signature

\section{Evaluation}
* simplicity/overhead for user (warning message?)
* how many apps can benefit from this?
* applicability
** do 3 case studies (examples of applying it to an application)
** do it on github
* How many additional lines of code required for using API?

* (performance (time+space))
* security of this scheme
** bits of security (e.g. signature size)
** assumptions/foundations (e.g. secure key distribution)
** key storage on Android phones can't be tampered with

* LIMITATIONS

* key add/revoke
** app perm to add
** app perm to remove
** done via syscalls, i.e. android holds this
*** some intent?
** some way for app developer to add centralized repo of keys
** automatic get key

* Does phone already have a pub key pair?
* backwards compatibility
** Read NDEF standards and how Android handles them

brainstorm                         works for                           bad
- NDEF record/type for signature   updated os
                                   app viewing finite \# records
- length exclused signature        above                               violates standards
  + above                          app viewing variable \# of records   

\section{Discussion}

\section{Future Work}
* user study, human testing
* overhead imposed by warnings/approvals
** delves into usable security
* compression/optimization
* OS perms, kernel mods to support this

\section{Conclusion}

\section{Acknowledgements}

\section{References}

\end{document}
