\documentclass[12pt]{article}

\usepackage[margin=1in]{geometry}

\begin{document}
\title{Defense Against W NFC}
\author{Max Feldman \and Stephanie Rogers \and Richard Xia}
\maketitle

\section{Abstract}
1-3 paragraph version of introduction

\section{Introduction}
1-2 page version of paper

As with any new technology, comes the potential for vulnerabilities within the system. Our paper attempts to analyze the coding practices and patterns at the application level of NFC applications. (intro?)

Related NFC Technology and Security - authenticating tags rather than users
* MiFare: hardware that allows the locking of data and allows symmetric mutual authentication of users and write protection to defend against reading and writing of the tag (privacy)
* FeliCa: authentication and write protection 
* AFAICT: lock the data (not overwritable), but no authentication!

\section{Background}
\subsection{NFC}

Near Field Communication (NFC) is an emerging technology for wireless communication which allows NFC-enabled devices to transfer small amounts of data at a close proximity, usually no more than a few centimeters. NFC offers current and predicted functionality of contactless transactions, data exchange and an alternative to traditional contactless technologies such as RFID and QR codes. Specifically aimed at mobile phones, devices have the ability to communicate with other NFC-enabled mobile phones to transfer data peer-to-peer in active mode or read information stored on NFC tags and cards in passive mode. 

\subsection{NFC with regards to the Android system}

The Android system sends and receives NFC data in the form of NDEF (NFC Data Exchange Format) messages. These messages, which are stored in the NFC tag, act as a container for one or more NDEF Records which contain typed data such as MIME-type media, a URI or a custom application payload (plagarism).  

Specifically, the Android system offers the ability to read and write NFC tags for passive mode, and beam NDEF messages from one device to another with Android Beam$^{2}$ for active mode. 

\subsection{Android Intents}
The core components of an Android application are activated through messages called intents.$^{1}$ Intents are used for inter-application and intra-application communication by sending messages with relevant information about operations to be performed to various applications.  

When NFC is enabled and the screen is unlocked, Android devices will always be searching for input from an NFC tag or device. When the device discovers an NFC tag, an intent is sent to the phone with information including which application is most appropriate to handle the data contained in the NFC tag. This information is processed by Android's special tag dispatch system to determine which activity to launch. It turns out that through NFC one can force some phones to parse images, videos, contacts, office documents, call arbitrary phone numbers and even open up web pages in the browser, all without user interaction.  

\subsection{Android Permission System}
Android applications run in a sandbox which allows areas of the system to be isolated and thus limit access to the system's resources. However, access to these security-related parts of the Android API can be granted to an application if the developer requests the appropriate permissions in the application's manifest. Android developers are expected to follow the idea of least-privilege with their permission requests and thus limit the permissions requested to only those that are absolutely necessary for the function and purpose of their application. By analyzing the use of permissions among applications, one can learn a lot about the potential vulnerabilities introduced in an application. 

1. http://developer.android.com/guide/components/intents-filters.html

2. http://developer.android.com/guide/topics/connectivity/nfc/nfc.html

\subsection{outline}
** characteristics of apps
*** tiny
*** superapps

* threat model
** writing sensitive data
** malicious tags which cause automatic running of apps which can cost money
** eavesdropping
** malware

\section{Related Work}

\section{Analysis of Apps}
* Why analyze
** general coding practices re: NFC
** what practices contribute to vulnerabilities of NFC apps
** As for as we know, no broad analysis of NFC app level stuff
** provide suggestion for NFC app devs (we are in a unique position in that this tech is up and coming)
* static
** trace where data read from NFC tags is ended
** other static tools?
** Beam
** Use of crypto libs
** Overwriting tags- do apps in general 0 out prior data, or not?

* current apps, what permissions do they use
<ML Classifier here>

* dynamic
** taintdroid
** droidblaze

* manual
** looked at source code
** played with apps
** trace exec paths if possible
** BEAM
*** If you scan an app, it'll take you to the app store
* permission scraping etc

\section{Proposed Solution}
* Goals/non-goals
* architecture
** distributed authentication
*** generalizable to arbitrary inputs to phone?

* Goals

As a user
I want to be able to verify a tag came from a given author
In order to ensure I am not acting on malicious inputs

As a user
I want to be able to keep track of tag authors
In order to expand the scope of tags I can trust

As a user
I want to be able to verify my own tags
In order to ensure use my own tags

As a user
I want to be able to encrypt my own tags
In order to ensure no one else can read my tags

As a user
I want to keep the identities of trusted tag authors correct
In case if good authors go bad

As a user
I want to report malicious tag authors
To warn other users of bad authors

As a tag author
I want to share my key with a user
To prevent others from impersonating me

** integrity
** authentication

* Non goals
** confidentiality of tag data
** ignoring malicious applications
** no central authority
** not ensuring security of key exchange


How to get keys?
* Go to website of tag author you trust, DL their key
* see it for the first time (good info in prompt?)
* add a key repo
* NFC tag (P2P sign)
* No signature: different prompt

Prompts:
* disable all pmopts
* never allow this key?
* allow key once
* allow key always
* allow all keys

* We should alway state why we make any choice

\section{Implementation}
* implementation
** proof of concept
** recommendations
* key file
* key add/remove
* warnings
* (a duplicate app w/ some functionality but no auth, for demo)
* API
* warning for invalid signature

\section{Evaluation}
* simplicity/overhead for user (warning message?)
* how many apps can benefit from this?
* applicability
** do 3 case studies (examples of applying it to an application)
** do it on github
* How many additional lines of code required for using API?

* (performance (time+space))
* security of this scheme
** bits of security (e.g. signature size)
** assumptions/foundations (e.g. secure key distribution)
** key storage on Android phones can't be tampered with

* LIMITATIONS

* key add/revoke
** app perm to add
** app perm to remove
** done via syscalls, i.e. android holds this
*** some intent?
** some way for app developer to add centralized repo of keys
** automatic get key

* Does phone already have a pub key pair?
* backwards compatibility
** Read NDEF standards and how Android handles them

brainstorm                         works for                           bad
- NDEF record/type for signature   updated os
                                   app viewing finite \# records
- length exclused signature        above                               violates standards
  + above                          app viewing variable \# of records   

\section{Discussion}

\section{Future Work}
* user study, human testing
* overhead imposed by warnings/approvals
** delves into usable security
* compression/optimization
* OS perms, kernel mods to support this

\section{Conclusion}

\section{Acknowledgements}

\section{References}

\end{document}
